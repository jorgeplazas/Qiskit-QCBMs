
\documentclass{beamer}

\usepackage[utf8]{inputenc}
%\usepackage[russian]{babel}

\logo{\includegraphics[width=.1\textwidth]{cat_2.png}}

\usepackage{euler}

\usetheme{Warsaw}
\definecolor{ufvblue}{RGB}{2, 53, 95}
\setbeamercolor{structure}{fg=ufvblue}
\setbeamercolor{title}{fg=ufvblue, bg=white}
\setbeamercolor{frametitle}{fg=ufvblue, bg=white}
\setbeamercolor{section in head/foot}{bg=ufvblue, fg=white}
\usecolortheme{owl}  % This is a dark theme color scheme

\usepackage{amsmath}
\usepackage{amssymb}
\usepackage{amsfonts}
\usepackage{amsthm}
\usepackage{braket}

\usepackage{verbatim}
%\usepackage[spanish]{babel}


\newcommand{\GL}{\mathrm{GL}}
\newcommand{\SL}{\mathrm{SL}}
\newcommand{\PGL}{\mathrm{PGL}}
\newcommand{\PSL}{\mathrm{PSL}}


\newcommand{\R}{\mathbb{R}}
\newcommand{\C}{\mathbb{C}}
\newcommand{\Q}{\mathbb{Q}}
\newcommand{\Z}{\mathbb{Z}}
\newcommand{\N}{\mathbb{N}}
\newcommand{\Hi}{\mathbb{H}}

\newcommand{\cA}{\mathcal{A}}
\newcommand{\cB}{\mathcal{B}}
\newcommand{\cF}{\mathcal{F}}
\newcommand{\cG}{\mathcal{G}}
\newcommand{\cH}{\mathcal{H}}
\newcommand{\cP}{\mathcal{P}}

\newcommand{\id}{\mathrm{I}}




% Information
\title[$ \ket{\psi}$]{Qiskit $\otimes$ QCBMs\\ {\,} \\QAMP 2026 Open House }
\date{ Quantum Circuit Born Machines in Qiskit }


\author[$\bra{\psi} QCHPC$]{{\large{Issue 64}}}




\titlegraphic{
	\includegraphics[width=0.15\textwidth]{Qiskit_image_0.png}
}

\begin{document}
	
\frame{\titlepage}



\begin{frame}{}
\begin{columns}
	% Column 1
	\begin{column}{0.55\textwidth}
		{\Large QAMP Open House} 	\vskip 0.3 truecm
		{\Large Qiskit $\otimes$ QCBMs} 
		\vskip 0.3 truecm
		{\Large{Issue 64}}\\
		\vskip 1.5 truecm
		{\small {\bf{Speakers: \\Debshata Choudhury \\ Jorge Plazas }} \\ (mentees)}\vskip 0.2 truecm
		
		
		{\small {\bf{Mentor: \\Natalie Hawkins }}}		
	\end{column}

	\begin{column}{0.5\textwidth}
		\begin{figure}
			\centering
			\includegraphics[width=\textwidth]{Qiskit_image_2.png}

		\end{figure}
	\end{column}
\end{columns}

\end{frame}

\begin{frame}{}
	\begin{columns}
		% Column 1
		\begin{column}{0.55\textwidth}
			{\Large Agenda } 
			
			\vskip 1.5 truecm
\begin{itemize}
	\item Quantum Circuit Born Machines. 
	\item Building blocks and auxiliary functions. 
	\item Experiments. 
	\item QCBMs for molecule generation. 
\end{itemize}			


		
		\end{column}
		
		\begin{column}{0.5\textwidth}
			\begin{figure}
				\centering
				\includegraphics[width=\textwidth]{Qiskit_image_1.png}
				
			\end{figure}
		\end{column}
	\end{columns}
	
\end{frame}


\begin{frame}{Quantum Circuit Born Machines}
	
	\begin{block}{}
		\vskip 0.4 truecm
		$\rightsquigarrow$ Hybrid intrinsic generative models that use quantum circuits to represent probability distributions. 
	\end{block}
	\vskip 0.4 truecm
	\begin{center}
		Distribution of data $\rightleftarrows$ Quantum state statistics
	\end{center}
	
	\begin{eqnarray*}
		\pi(X) &\sim&  |\braket{X | \psi_{\theta}}|^2
	\end{eqnarray*}
	
	%	\includegraphics[width=.5\textwidth]{Quantum_Computer.jpg}
	
\end{frame}

\begin{frame}{QCBMs - Building blocks}
	

The {\emph{circuit}} part of a QCBM consist of two kinds of layers:
	\begin{itemize}
		\item Rotation layers.
		\item Entanglement layers.
	\end{itemize}
This structure can be seen as an instance of  the $$\mathrm{efficient\_su2}$$ ansatz in qiskit.  	

State statistics are obtained via the  $$\mathrm{Sampler}$$ primitive. 

\end{frame}

\begin{frame}{QCBMs - Functions for training}

\begin{block}{Kernels and similarity measures}
In order to train the models various kernels and loss functions were implemented. These were used measure the discrepancy between sampled distributions and target distributions.   

\pause 
	\begin{itemize}
		\item Radial Basis Function (RBF) Kernel
		\item Squared Maximum Mean Discrepancy (MMD) Loss.
		\item Kullback–Leibler (KL) Divergence.
	\end{itemize} 
 \end{block}
	
\end{frame}


\begin{frame}{Discrete Gaussian experiment}
	
Sum of Three Dice (Discrete Gaussian) Experiment
	\begin{columns}
		% Column 1
		\begin{column}{0.55\textwidth}
			\begin{itemize}
			\item Fidelity = 98.7\% 
			
			Generated distribution closely matches the target distribution.
			\item Sinkhorn OT = 0.0925
			
			Negligible probability mass movement; near-perfect distribution alignment. 
		\end{itemize} 	
		\end{column}
		
	\begin{column}{0.5\textwidth}
			\begin{figure}
				\centering
				\includegraphics[width=\textwidth]{Gaussian2}
				
			\end{figure}
		\end{column}
	\end{columns}
	
\end{frame}


\begin{frame}{Discrete Gaussian experiment}
	\begin{columns}
		% Column 1
		\begin{column}{0.55\textwidth}
			\begin{itemize}
				\item KL Divergence (low) 
				
	
	 Minimal information loss between true and learned distributions. 
			
			\item MMD (low) 
			
			Strong sample-level similarity; effective global distribution matching.
			\end{itemize} 	
		\end{column}
		
		\begin{column}{0.5\textwidth}
			\begin{figure}
				\centering
				\includegraphics[width=.7\textwidth]{GaussianKLDiv}
				
				
				\includegraphics[width=.7\textwidth]{GaussianMMDLoss}
			\end{figure}
		\end{column}
	\end{columns}
	
\end{frame}

\begin{frame}{Bars \& Stripes (BAS) Experiment}
	
	\begin{columns}
		% Column 1
		\begin{column}{0.55\textwidth}
			\begin{itemize}
			\item Fidelity = 72.7\% 
			
			Partial match with the target BAS distribution. 
			
			\item Sinkhorn OT = 0.4456 
			
			Moderate probability mass movement; incomplete alignment with BAS modes.
			\end{itemize} 	
		\end{column}
		
		\begin{column}{0.5\textwidth}
			\begin{figure}
				\centering
				\includegraphics[width=\textwidth]{BAS2}
				
			\end{figure}
		\end{column}
	\end{columns}
	
\end{frame}


\begin{frame}{Bars \& Stripes (BAS) Experiment}
	\begin{columns}
		% Column 1
		\begin{column}{0.55\textwidth}
			\begin{itemize}
				\item  KL Divergence (moderate)
				
				Noticeable divergence due to probability leakage into invalid patterns.
				
				\item MMD (moderate)
				
				Structural similarity present, but mode precision remains imperfect.
			\end{itemize} 	
		\end{column}
		
		\begin{column}{0.5\textwidth}
			\begin{figure}
				\centering
				\includegraphics[width=.7\textwidth]{BASKLDiv}
				
				
				\includegraphics[width=.7\textwidth]{BASMMDLoss}
			\end{figure}
		\end{column}
	\end{columns}
	
\end{frame}


\begin{frame}{Chemical space exploration and molecule generation}
	\begin{block}{Our project was partly inspired by }
	
	
	{\emph{Quantum-computing-enhanced algorithm unveils potential KRAS inhibitors.}} by Ghazi Vakili, M., Gorgulla, C., Snider, J. et al.  
	
	[Nature Biotechnology (2025)]
	\end{block}
	
Where QCBMs are used as a component in a hybrid generative model used for the generation of novel KRAS inhibitors\footnote{KRAS is a common and hard-to-treat oncogene in many solid cancers.}.	
	
\end{frame}

\begin{frame}{Chemical space exploration and molecule generation \\ (After [Aspuru-Guzik et al.])}

		We used STONED-SELFIES in conjunction with QCBMs to learn the probability distribution, within the chemical space, of neighboring KRAS inhibitor molecules.

	\includegraphics[width=.6\textwidth]{Molecule}
\end{frame}




\begin{frame}{}
	\begin{columns}
		% Column 1
		\begin{column}{0.55\textwidth}

	\begin{figure}
		\centering
		\includegraphics[width=.6\textwidth]{Qiskit_Image_0}
	\end{figure}

\end{column}
			\begin{column}{0.5\textwidth}
	
				
	{\Huge{THANKS}}
	
	to the QAMP community. 
	
	\vskip 0.6 truecm
	
	{\Large{$\oplus$ SPECIAL THANKS}}
	
	\vskip 0.2 truecm
	to Astri, Radha and our amazing mentor Natalie Hawkins.		
\vskip 1.6 truecm

Visit our repo at:
	
	\url{https://github.com/QCHPC/qiskit_QCBMs}
		\end{column}
	\end{columns}
	
\end{frame}


\end{document}

