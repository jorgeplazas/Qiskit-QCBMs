%%%%%%%%%%%%%%%%%%%%%%%%%%%%%%%%%%%%%%%%%
% Debshata Choudhury
% Natalie Hawkins
% Jorge Plazas
%
% Qiskit QCBMs
% 
%%%%%%%%%%%%%%%%%%%%%%%%%%%%%%%%%%%%%%%%%%




\documentclass[a4paper, 10pt]{article}
\usepackage[utf8]{inputenc}
\usepackage[T1]{fontenc}
\usepackage{lmodern}
\usepackage{amsfonts,amssymb}
\usepackage{amssymb,amsmath,amsthm}
\usepackage[pdftex]{hyperref}
\usepackage{braket}



\title {Qiskit QCBMs \\Checkpoint 2 report}
\author{Debshata Choudhury, \and Natalie Hawkins, \and Jorge Plazas}


\begin{document}
\maketitle

\begin{abstract}
\noindent We describe our progress in {\emph{Qiskit QCBMs}}, a project where we develop a set of tools for the implementation of Quantum Circuit Born Machines within the Qiskit 2.x and Qiskit Patterns framework. This project is part of Qiskit Advocate Mentorship Program QAMP 2025.  
\end{abstract}
	
\section{The project in a Nutshell}

Quantum Circuit Born Machines (QCBMs) are generative models in which the probability distribution of a dataset is represented in terms of a quantum state via Born's rule. QCBMs are known for their expressive power and have been widely recognized as a useful component in hybrid algorithms with a potential for quantum advantage. 


Our aim is to take advantage of various tools in the Qiskit ecosystem in order to optimize QCBMs for their use in near term devises.  Qiskit QCBMs is being developed as a qiskit add-on which will be accompanied by a set of notebooks both showcasing the tools through examples and serving as an introduction to the theory. 



\section{Workflow, progress and status of the project}

Our team, consisting of two mentees together with one mentor, has held weekly meetings since the beginning of the program in October 2025. 

At this point we have:

\begin{itemize}

	
	\item Build a github repository where the code developed so far is hosted. The repo also contains a substantial amount of information on QCBMs and documents the evolution of the project:\\
	\url{https://github.com/jorgeplazas/Qiskit-QCBMs}
	
	\item Reviewed the state of the art both in the literature and in avaliable projects which use QCBMs. 
	
	\item Implemented a set of core and auxiliary functions.
	
	\item Ran various experiments and tests using simulators.
	
	\item Made first versions of the jupyter notebooks accompanying the project.  
	
	\item Structured some of the tools within a documented module. 
	
\end{itemize}

\subsection*{A word on Qiskit}
The circuit part of a QCBM consist of two kinds of layers:
\begin{itemize}
	\item Rotation layers.
	\item Entanglement layers. 
\end{itemize}
This structure can be realized as an instance of  the \texttt{efficient su2}\footnote{\url{https://quantum.cloud.ibm.com/docs/en/api/qiskit/qiskit.circuit.library.EfficientSU2}}  circuit in qiskit.  Because of the above many of the choices made in the project reflect those underlying this type of circuit as the core component. 


The training loop of the model consist of a series of passes where samples from states prepared by the circuit are taken. Because of this the  \texttt{Sampler}\footnote{\url{https://quantum.cloud.ibm.com/docs/en/api/qiskit-ibm-runtime/sampler-v2}} primitive plays a central role in the project. 


\subsection*{Astrolabe}

Three works have been specially relevant in guiding the route taken by the project:

\begin{enumerate}
	\item The seminal work of Liu and Wang \cite{LiuWang2018} where QCBMs are studied in depth.
	\item The work ok Hamilton et. al. \cite{Hamilton2019} which building upon the above work uses QCBMs to benchmark quantum computers. 
	\item The work of Ghazi Vakili et. al. \cite{GhaziVakili2025} where new inhibitors for KRAS are discovered using a hybrid algorithm with a QCBM as its main quantum component. 
\end{enumerate}


From the first two we have taken our base-test examples for distributions to be learned by the models: Gaussian distributions and the distribution underlying the Bars and Stripes dataset. To these we have added the distribution of a sum-of-dice simulator. 

The last reference was one of our initial motivations and contains a pipeline that we aim to improve.   
	 


 

\section{Challenges}

We are currently working towards a pipeline for testing our tools in real quantum hardware. Optimizing this for the resources that are available to us is for now the biggest challenge facing the project.  





\section{The path forward}

Our work in the coming months will focus on four core aspects: 

\begin{enumerate}
	\item Run experiments and tests in quantum hardware.
	\item Showcase the tools by implementing a model similar to the one used in \cite{GhaziVakili2025}.
	\item Structure the tools developed as a quiskit add-on. 
	\item Write all the relevant documentation.  
\end{enumerate}




\bibliographystyle{unsrt}
\bibliography{Qiskit-QCBM.bib}


\end{document}
